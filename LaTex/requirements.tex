

\section{Requirements}

\subsection{Motivation}
The original idea was to develop an App that can be used to identify certain areas in cities where users feel afraid or, in a more general term, not safe.\\
In the process of developing the app the use case was extended to not only give users the ability to mark places as unsafe but also give the option of marking a place as positive. The process lead to the app called Feedbacker.

\subsection{User groups}
The main targeted user group are individual people. Users have the option to see feedback of other users in order to see how different areas are perceived, or to get ideas what to do/where to go while planning a trip. \\
City administrations might also have interest in the gathered data to identify urban areas that could be developed further, and to determine what makes certain regions stand out in comparison to other regions.

\subsection{User Actions}
\subsubsection{Sending of Feedback}
The user has the option of sending two kinds of Feedback, positive and negative. After the user has sent the feedback it will be saved to the database containing the user's current location, the current time and the kind of feedback.

\subsubsection{Editing of Feedback}
After sending the Feedback the user has the ability to further specify it. \\
One of the main points of editing is categorizing the Feedback. The available categories are:
\begin{table}[h]
\begin{center}
\begin{tabular}{l  l}
positive & negative \\
\hline
  Places to sit & Dark Place \\
  Public/Clean Toilets & Dirty \\
  Disablity Access & Shady Group of People \\
  Public Wi-Fi & Not pedestrian friendly \\
  Place to Eat/Drink & Graffiti \\
  Nice View & \\
\end{tabular}
 \caption{Categories}\label{tab_categories}
 \label{table:categories}
\end{center}
\end{table}

To give additional information about the feedback or the place of the feedback the user can write a descriptive text. If the user decides to mark the feedback as public, other users are able to see the feedback on a map. Users can also change their feedback from positive to negative (or vice versa), for example when a place was altered or if the user clicked the wrong button when sending his feedback.

\subsection{Other Apps \& Services}
There are other apps and services which have some similarities to Feedbacker. In the following I show similarities and differences to two of those Apps.

\paragraph{Google Maps}
Google Maps has a lot of information on restaurants, shops and places in general. Users can not only rate and comment on places but Google also the option for users to add new places and enhance information like opening hours of existing places \cite{gMaps}.\\
The information is however not as prominent and readily available as it is with Feedbacker, because in Google Maps the feedback of other users is not visible at first glance.

\paragraph{wheelmap.org}
"Wheelmap is a map for finding wheelchair accessible places. The map works similar to Wikipedia: anyone can contribute and mark public places around the world according to their wheelchair accessibility. The criteria for marking places is based on a simple traffic light system". It distinguishes between 130 types of places \cite{wheel}.\\
Similar to Feedbacker, wheelmap.org shows user-submitted ratings of places directly on a map and it is also visible whether a place is rated as good or bad.\\
With Feedbacker however, users can rate places in more ways than just specialized on one feature such as wheelchair access.
