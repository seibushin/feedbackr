\section{Conclusion}
This work documents the process of developing a multiplatform, location-based app called Feedbacker.\\
\paragraph{Development Process}
The development process showed that Firebase is a great tool not only for developing an app with both Android and iOS support in general, but also for achieving this quickly and efficiently.\\
Especially when working with database-focused apps like Feedbacker Firebase can play out its strengths. The simplicity of
While there are still many things that could be improved in the future, especially regarding the user interface in  the iOS version, this work showed that it is possible to develop functioning prototypes for two platforms in a relatively short amount of time.

\paragraph{Future}
The future of the app depends heavily on its success after being released in the Apple App Store and Google Play Store respectively.\\
The success should not only be measured by the number of installations but especially by usage numbers. As Feedbacker is a crowdsourcing app it relies on active users that send feedback regularly. It is also important that the group of active users is as big as possible. The value of the app might increase further if it is used by many people in the same region, thus resulting in a large density of feedbacks in that region. This effect might be achieved by investing in advertising the app in selected cities and communities.\\
